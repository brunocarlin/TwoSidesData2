\PassOptionsToPackage{unicode=true}{hyperref} % options for packages loaded elsewhere
\PassOptionsToPackage{hyphens}{url}
%
\documentclass[]{article}
\usepackage{lmodern}
\usepackage{amssymb,amsmath}
\usepackage{ifxetex,ifluatex}
\usepackage{fixltx2e} % provides \textsubscript
\ifnum 0\ifxetex 1\fi\ifluatex 1\fi=0 % if pdftex
  \usepackage[T1]{fontenc}
  \usepackage[utf8]{inputenc}
  \usepackage{textcomp} % provides euro and other symbols
\else % if luatex or xelatex
  \usepackage{unicode-math}
  \defaultfontfeatures{Ligatures=TeX,Scale=MatchLowercase}
\fi
% use upquote if available, for straight quotes in verbatim environments
\IfFileExists{upquote.sty}{\usepackage{upquote}}{}
% use microtype if available
\IfFileExists{microtype.sty}{%
\usepackage[]{microtype}
\UseMicrotypeSet[protrusion]{basicmath} % disable protrusion for tt fonts
}{}
\IfFileExists{parskip.sty}{%
\usepackage{parskip}
}{% else
\setlength{\parindent}{0pt}
\setlength{\parskip}{6pt plus 2pt minus 1pt}
}
\usepackage{hyperref}
\hypersetup{
            pdftitle={exploratory data analysis: basics Python part 2},
            pdfborder={0 0 0},
            breaklinks=true}
\urlstyle{same}  % don't use monospace font for urls
\usepackage[margin=1in]{geometry}
\usepackage{color}
\usepackage{fancyvrb}
\newcommand{\VerbBar}{|}
\newcommand{\VERB}{\Verb[commandchars=\\\{\}]}
\DefineVerbatimEnvironment{Highlighting}{Verbatim}{commandchars=\\\{\}}
% Add ',fontsize=\small' for more characters per line
\usepackage{framed}
\definecolor{shadecolor}{RGB}{248,248,248}
\newenvironment{Shaded}{\begin{snugshade}}{\end{snugshade}}
\newcommand{\AlertTok}[1]{\textcolor[rgb]{0.94,0.16,0.16}{#1}}
\newcommand{\AnnotationTok}[1]{\textcolor[rgb]{0.56,0.35,0.01}{\textbf{\textit{#1}}}}
\newcommand{\AttributeTok}[1]{\textcolor[rgb]{0.77,0.63,0.00}{#1}}
\newcommand{\BaseNTok}[1]{\textcolor[rgb]{0.00,0.00,0.81}{#1}}
\newcommand{\BuiltInTok}[1]{#1}
\newcommand{\CharTok}[1]{\textcolor[rgb]{0.31,0.60,0.02}{#1}}
\newcommand{\CommentTok}[1]{\textcolor[rgb]{0.56,0.35,0.01}{\textit{#1}}}
\newcommand{\CommentVarTok}[1]{\textcolor[rgb]{0.56,0.35,0.01}{\textbf{\textit{#1}}}}
\newcommand{\ConstantTok}[1]{\textcolor[rgb]{0.00,0.00,0.00}{#1}}
\newcommand{\ControlFlowTok}[1]{\textcolor[rgb]{0.13,0.29,0.53}{\textbf{#1}}}
\newcommand{\DataTypeTok}[1]{\textcolor[rgb]{0.13,0.29,0.53}{#1}}
\newcommand{\DecValTok}[1]{\textcolor[rgb]{0.00,0.00,0.81}{#1}}
\newcommand{\DocumentationTok}[1]{\textcolor[rgb]{0.56,0.35,0.01}{\textbf{\textit{#1}}}}
\newcommand{\ErrorTok}[1]{\textcolor[rgb]{0.64,0.00,0.00}{\textbf{#1}}}
\newcommand{\ExtensionTok}[1]{#1}
\newcommand{\FloatTok}[1]{\textcolor[rgb]{0.00,0.00,0.81}{#1}}
\newcommand{\FunctionTok}[1]{\textcolor[rgb]{0.00,0.00,0.00}{#1}}
\newcommand{\ImportTok}[1]{#1}
\newcommand{\InformationTok}[1]{\textcolor[rgb]{0.56,0.35,0.01}{\textbf{\textit{#1}}}}
\newcommand{\KeywordTok}[1]{\textcolor[rgb]{0.13,0.29,0.53}{\textbf{#1}}}
\newcommand{\NormalTok}[1]{#1}
\newcommand{\OperatorTok}[1]{\textcolor[rgb]{0.81,0.36,0.00}{\textbf{#1}}}
\newcommand{\OtherTok}[1]{\textcolor[rgb]{0.56,0.35,0.01}{#1}}
\newcommand{\PreprocessorTok}[1]{\textcolor[rgb]{0.56,0.35,0.01}{\textit{#1}}}
\newcommand{\RegionMarkerTok}[1]{#1}
\newcommand{\SpecialCharTok}[1]{\textcolor[rgb]{0.00,0.00,0.00}{#1}}
\newcommand{\SpecialStringTok}[1]{\textcolor[rgb]{0.31,0.60,0.02}{#1}}
\newcommand{\StringTok}[1]{\textcolor[rgb]{0.31,0.60,0.02}{#1}}
\newcommand{\VariableTok}[1]{\textcolor[rgb]{0.00,0.00,0.00}{#1}}
\newcommand{\VerbatimStringTok}[1]{\textcolor[rgb]{0.31,0.60,0.02}{#1}}
\newcommand{\WarningTok}[1]{\textcolor[rgb]{0.56,0.35,0.01}{\textbf{\textit{#1}}}}
\usepackage{graphicx,grffile}
\makeatletter
\def\maxwidth{\ifdim\Gin@nat@width>\linewidth\linewidth\else\Gin@nat@width\fi}
\def\maxheight{\ifdim\Gin@nat@height>\textheight\textheight\else\Gin@nat@height\fi}
\makeatother
% Scale images if necessary, so that they will not overflow the page
% margins by default, and it is still possible to overwrite the defaults
% using explicit options in \includegraphics[width, height, ...]{}
\setkeys{Gin}{width=\maxwidth,height=\maxheight,keepaspectratio}
\setlength{\emergencystretch}{3em}  % prevent overfull lines
\providecommand{\tightlist}{%
  \setlength{\itemsep}{0pt}\setlength{\parskip}{0pt}}
\setcounter{secnumdepth}{0}
% Redefines (sub)paragraphs to behave more like sections
\ifx\paragraph\undefined\else
\let\oldparagraph\paragraph
\renewcommand{\paragraph}[1]{\oldparagraph{#1}\mbox{}}
\fi
\ifx\subparagraph\undefined\else
\let\oldsubparagraph\subparagraph
\renewcommand{\subparagraph}[1]{\oldsubparagraph{#1}\mbox{}}
\fi

% set default figure placement to htbp
\makeatletter
\def\fps@figure{htbp}
\makeatother

\usepackage{etoolbox}
\makeatletter
\providecommand{\subtitle}[1]{% add subtitle to \maketitle
  \apptocmd{\@title}{\par {\large #1 \par}}{}{}
}
\makeatother

\title{exploratory data analysis: basics Python part 2}
\providecommand{\subtitle}[1]{}
\subtitle{Python part 2}
\author{}
\date{\vspace{-2.5em}2020-01-25}

\begin{document}
\maketitle

Let's see what version of python this env is running.

\begin{Shaded}
\begin{Highlighting}[]
\NormalTok{reticulate}\OperatorTok{::}\KeywordTok{py_config}\NormalTok{()}
\end{Highlighting}
\end{Shaded}

\begin{verbatim}
## python:         /home/bruno-carlin/Documents/GIthub/TwoSidesData2/.venv/bin/python
## libpython:      /opt/python/3.7/lib/libpython3.7m.so
## pythonhome:     /opt/python/3.7:/opt/python/3.7
## virtualenv:     /home/bruno-carlin/Documents/GIthub/TwoSidesData2/.venv/bin/activate_this.py
## version:        3.7.4 (default, Aug 13 2019, 20:35:49)  [GCC 7.3.0]
## numpy:          /home/bruno-carlin/Documents/GIthub/TwoSidesData2/.venv/lib/python3.7/site-packages/numpy
## numpy_version:  1.18.1
## 
## NOTE: Python version was forced by use_python function
\end{verbatim}

\begin{Shaded}
\begin{Highlighting}[]
\ImportTok{import}\NormalTok{ numpy }\ImportTok{as}\NormalTok{ np}
\ImportTok{import}\NormalTok{ matplotlib.pyplot }\ImportTok{as}\NormalTok{ plt}
\ImportTok{import}\NormalTok{ seaborn }\ImportTok{as}\NormalTok{ sns}
\ImportTok{import}\NormalTok{ pandas }\ImportTok{as}\NormalTok{ pd}
\ImportTok{import}\NormalTok{ os}
\end{Highlighting}
\end{Shaded}

\hypertarget{second-post}{%
\section{Second Post}\label{second-post}}

\hypertarget{objectives}{%
\subsection{Objectives}\label{objectives}}

\hypertarget{define-the-variables-used-in-the-conclusion}{%
\subsection{Define the variables used in the
conclusion}\label{define-the-variables-used-in-the-conclusion}}

In our case, we choose to use
\protect\hyperlink{python_hypothesis_testing}{salary \textasciitilde{}
sex,region} region was added to test whether
\href{https://en.wikipedia.org/wiki/Simpson\%27s_paradox}{Simpson's
paradox} was at play.

\hypertarget{using-masks-or-other-methods-to-filter-the-data}{%
\subsection{Using masks or other methods to filter the
data}\label{using-masks-or-other-methods-to-filter-the-data}}

This objective was mostly done using the groupby function.

\hypertarget{visualizing-the-hypothesis}{%
\subsection{Visualizing the
hypothesis}\label{visualizing-the-hypothesis}}

\protect\hyperlink{python_plot_histograms}{We were advised to use two
histograms combined to get a preview of our answer.}

\hypertarget{conclusion}{%
\subsection{Conclusion}\label{conclusion}}

Comment on our findings.

\hypertarget{before-we-start}{%
\subsection{Before we start}\label{before-we-start}}

\hypertarget{reservations}{%
\subsubsection{Reservations}\label{reservations}}

This is an exercise where we were supposed to ask a relevant question
using the data from the IBGE(Brazil's main data collector) database of
1970.

Our group decided to ask whether women received less than man, we
expanded the analysis hoping to avoid the Simpson's paradox.

This is just an basic inference, and it's results are therefore only
used for studying purposes I don't believe any finding would be relevant
using just this approach but some basic operations can be used in a more
impact full work.

\hypertarget{data-dictionary}{%
\subsubsection{Data Dictionary}\label{data-dictionary}}

We got a Data Dictionary that will be very useful for our Analysis, it
contains all the required information about the encoding of the columns
and the intended format that the folks at STATA desired.

Portuguese

Descrição do Registro de Indivíduos nos EUA.

Dataset do software STATA (pago), vamos abri-lo com o pandas e
transforma-lo em DataFrame.

Variável 1 -- CHAVE DO INDIVÍDUO ? Formato N - Numérico ? Tamanho 11
dígitos (11 bytes) ? Descrição Sumária Identifica unicamente o indivíduo
na amostra.

Variável 2 - IDADE CALCULADA EM ANOS ? Formato N - Numérico ? Tamanho 3
dígitos (3 bytes) ? Descrição Sumária Identifica a idade do morador em
anos completos.

Variável 3 -- SEXO ? Formato N - Numérico ? Tamanho 1 dígito (1 byte) ?
Quantidade de Categorias 3 ? Descrição Sumária Identifica o sexo do
morador. Categorias (1) homem, (2) mulher e (3) gestante.

Variável 4 -- ANOS DE ESTUDO ? Formato N - Numérico ? Tamanho 2 dígitos
(2 bytes) ? Quantidade de Categorias 11 ? Descrição Sumária Identifica o
número de anos de estudo do morador. Categorias (05) Cinco ou menos,
(06) Seis, (07) Sete, (08) Oito, (09) Nove, (10) Dez, (11) Onze, (12)
Doze, (13) Treze, (14) Quatorze, (15) Quinze ou mais.

Variável 5 -- COR OU RAÇA ? Formato N - Numérico ? Tamanho 2 dígitos (2
bytes) ? Quantidade de Categorias 6 ? Descrição Sumária Identifica a Cor
ou Raça declarada pelo morador. Categorias (01) Branca, (02) Preta, (03)
Amarela, (04) Parda, (05) Indígena e (09) Não Sabe.

Variável 6 -- VALOR DO SALÁRIO (ANUALIZADO) ? Formato N - Numérico ?
Tamanho 8 dígitos (8 bytes) ? Quantidade de Decimais 2 ? Descrição
Sumária Identifica o valor resultante do salário anual do indivíduo.
Categorias especiais (-1) indivíduo ausente na data da pesquisa e
(999999) indivíduo não quis responder.

Variável 7 -- ESTADO CIVIL ? Formato N - Numérico ? Tamanho 1 dígito (1
byte) ? Quantidade de Categorias 2 ? Descrição Sumária Dummy que
identifica o estado civil declarado pelo morador. Categorias (1) Casado,
(0) não casado.

Variável 8 -- REGIÃO GEOGRÁFICA ? Formato N - Numérico ? Tamanho 1
dígito (1 byte) ? Quantidade de Categorias 5 ? Descrição Sumária
Identifica a região geográfica do morador. Categorias (1) Norte, (2)
Nordeste, (3) Sudeste, (4) Sul e (5) Centro-oeste.

English

Description of the US Individual Registry.

Dataset of the STATA software (paid), we will open it with pandas and
turn it into DataFrame.

Variable 1 - KEY OF THE INDIVIDUAL? Format N - Numeric? Size 11 digits
(11 bytes)? Summary Description Uniquely identifies the individual in
the sample.

Variable 2 - AGE CALCULATED IN YEARS? Format N - Numeric? Size 3 digits
(3 bytes)? Summary Description Identifies the age of the resident in
full years.

Variable 3 - SEX? Format N - Numeric? Size 1 digit (1 byte)? Number of
Categories 3? Summary Description Identifies the gender of the resident.
Categories (1) men, (2) women and (3) pregnant women.

Variable 4 - YEARS OF STUDY? Format N - Numeric? Size 2 digits (2
bytes)? Number of Categories 11? Summary Description Identifies the
number of years of study of the resident. Categories (05) Five or less,
(06) Six, (07) Seven, (08) Eight, (09) Nine, (10) Dec, (11) Eleven, (12)
Twelve, (13) Thirteen, (14 ) Fourteen, (15) Fifteen or more.

Variable 5 - COLOR OR RACE? Format N - Numeric? Size 2 digits (2 bytes)?
Number of Categories 6? Summary Description Identifies the Color or Race
declared by the resident. Categories (01) White, (02) Black, (03)
Yellow, (04) Brown, (05) Indigenous and (09) Don't know.

Variable 6 - WAGE VALUE (ANNUALIZED)? Format N - Numeric? Size 8 digits
(8 bytes)? Number of decimals 2? Summary Description Identifies the
amount resulting from the individual's annual salary. Special categories
(-1) individual absent on the survey date and (999999) individual did
not want to answer.

Variable 7 - CIVIL STATE? Format N - Numeric? Size 1 digit (1 byte)?
Number of Categories 2? Summary Description Dummy that identifies the
marital status declared by the resident. Categories (1) Married, (0) Not
married.

Variable 8 - GEOGRAPHICAL REGION? Format N - Numeric? Size 1 digit (1
byte)? Number of Categories 5? Summary Description Identifies the
resident's geographic region. Categories (1) North, (2) Northeast, (3)
Southeast, (4) South and (5) Midwest.

\hypertarget{python}{%
\section{Python}\label{python}}

\hypertarget{importing-the-dataset-from-part-1}{%
\subsection{Importing the dataset from part
1}\label{importing-the-dataset-from-part-1}}

\begin{Shaded}
\begin{Highlighting}[]
\NormalTok{df_sex_thesis }\OperatorTok{=}\NormalTok{pd.read_feather(r.file_path_linux }\OperatorTok{+} \StringTok{'/sex_thesis_assignment.feather'}\NormalTok{)}
\end{Highlighting}
\end{Shaded}

\begin{Shaded}
\begin{Highlighting}[]
\NormalTok{df_sex_thesis.info()}
\end{Highlighting}
\end{Shaded}

\begin{verbatim}
## <class 'pandas.core.frame.DataFrame'>
## RangeIndex: 65795 entries, 0 to 65794
## Data columns (total 9 columns):
## index           65795 non-null int64
## age             65795 non-null int64
## sex             65795 non-null object
## years_study     65795 non-null category
## color_race      65795 non-null object
## salary          65795 non-null float64
## civil_status    65795 non-null object
## region          65795 non-null object
## log_salary      65795 non-null float64
## dtypes: category(1), float64(2), int64(2), object(4)
## memory usage: 4.1+ MB
\end{verbatim}

Let's get going first we need to define which variables we will add to
the hypothesis to isolate the factor of salary \textasciitilde{} sex, a
simple way of doing that is that if we consider that our sample of
individuals is random in nature we could simple compare the means of the
individuals given their sex and see if there is a significant difference
in their means, we can calculate that by using the standard deviation of
the calculated means.

A good graphic to get an idea if these effects would be significant was
the bar plots we used in part 1.

But if we are working with Categorical variable we can use a groupby
aproach to glimpse at the difference in means

I am using the log salary feature from the post 1

\begin{Shaded}
\begin{Highlighting}[]
\NormalTok{df_sex_thesis.groupby(}\StringTok{'sex'}\NormalTok{).mean().log_salary}
\end{Highlighting}
\end{Shaded}

\begin{verbatim}
## sex
## man      9.026607
## woman    8.607023
## Name: log_salary, dtype: float64
\end{verbatim}

Remember that in order to transform back our log variables you can do
\(a^2 + b^2 = c^2 + D\) for our example 4 ok

\[\text{softmax}(z)_j = \frac{e^{z_j}}{\sum_i^{\text{dim}(z)}e^{z_i}}\]

Group by explanation

Group by in pandas is a method that accepts a list of elements in this
case just `sex' and applies consequent operation in each group, in this
case we applied the mean method from a pandas DataFrame, we used .salary
to return just the mean for the salary variable.

\begin{Shaded}
\begin{Highlighting}[]
\NormalTok{df_sex_thesis.groupby([}\StringTok{'sex'}\NormalTok{,}\StringTok{'region'}\NormalTok{]).mean().salary}
\end{Highlighting}
\end{Shaded}

\begin{verbatim}
## sex    region   
## man    midwest      15263.832776
##        north        11645.335382
##        south        15507.600528
##        southeast    14755.538089
## woman  midwest      11394.125688
##        north         9932.078329
##        northeast    24737.436554
##        south        10159.399214
##        southeast    10826.777483
## Name: salary, dtype: float64
\end{verbatim}

Adding standard deviations

\begin{Shaded}
\begin{Highlighting}[]
\NormalTok{df_sex_thesis.groupby([}\StringTok{'sex'}\NormalTok{]).std().salary}
\end{Highlighting}
\end{Shaded}

\begin{verbatim}
## sex
## man      21832.256493
## woman    11949.880892
## Name: salary, dtype: float64
\end{verbatim}

These are really big Standard deviations! Remembering from stats that +2
SD's gives about a 95\% confidence interval we are not even close.

To add both togheter we can use pandas agg method

\begin{Shaded}
\begin{Highlighting}[]
\NormalTok{df_sex_thesis.groupby([}\StringTok{'sex'}\NormalTok{]).agg([}\StringTok{'mean'}\NormalTok{,}\StringTok{'std'}\NormalTok{]).salary}
\end{Highlighting}
\end{Shaded}

\begin{verbatim}
##                mean           std
## sex                              
## man    14302.491879  21832.256493
## woman  10642.502734  11949.880892
\end{verbatim}

Now we can calculate boundaries

\begin{Shaded}
\begin{Highlighting}[]
\NormalTok{df_agg }\OperatorTok{=}\NormalTok{df_sex_thesis.groupby([}\StringTok{'sex'}\NormalTok{]).agg([}\StringTok{'mean'}\NormalTok{,}\StringTok{'std'}\NormalTok{]).salary}
\end{Highlighting}
\end{Shaded}

\begin{Shaded}
\begin{Highlighting}[]
\NormalTok{df_agg}
\end{Highlighting}
\end{Shaded}

\begin{verbatim}
##                mean           std
## sex                              
## man    14302.491879  21832.256493
## woman  10642.502734  11949.880892
\end{verbatim}

\begin{Shaded}
\begin{Highlighting}[]
\NormalTok{df_agg[}\StringTok{'lower_bound'}\NormalTok{] }\OperatorTok{=}\NormalTok{ df_agg[}\StringTok{'mean'}\NormalTok{] }\OperatorTok{-}\NormalTok{ df_agg[}\StringTok{'std'}\NormalTok{] }\OperatorTok{*} \DecValTok{2}
\NormalTok{df_agg[}\StringTok{'upper_bound'}\NormalTok{] }\OperatorTok{=}\NormalTok{ df_agg[}\StringTok{'mean'}\NormalTok{] }\OperatorTok{+}\NormalTok{ df_agg[}\StringTok{'std'}\NormalTok{] }\OperatorTok{*} \DecValTok{2}
\end{Highlighting}
\end{Shaded}

\begin{Shaded}
\begin{Highlighting}[]
\NormalTok{df_agg}
\end{Highlighting}
\end{Shaded}

\begin{verbatim}
##                mean           std   lower_bound   upper_bound
## sex                                                          
## man    14302.491879  21832.256493 -29362.021106  57967.004865
## woman  10642.502734  11949.880892 -13257.259050  34542.264519
\end{verbatim}

Cool but to verbose to be repeated we can easily make this series of
operation into a function

\begin{Shaded}
\begin{Highlighting}[]
\KeywordTok{def}\NormalTok{ groupby_bound(df,groupby_variables,value_variables):}
\NormalTok{  df_agg }\OperatorTok{=}\NormalTok{  df.groupby(groupby_variables).agg([}\StringTok{'mean'}\NormalTok{,}\StringTok{'std'}\NormalTok{])[value_variables]}
\NormalTok{  df_agg[}\StringTok{'lower_bound'}\NormalTok{] }\OperatorTok{=}\NormalTok{ df_agg[}\StringTok{'mean'}\NormalTok{] }\OperatorTok{-}\NormalTok{ df_agg[}\StringTok{'std'}\NormalTok{] }\OperatorTok{*} \DecValTok{2}
\NormalTok{  df_agg[}\StringTok{'upper_bound'}\NormalTok{] }\OperatorTok{=}\NormalTok{ df_agg[}\StringTok{'mean'}\NormalTok{] }\OperatorTok{+}\NormalTok{ df_agg[}\StringTok{'std'}\NormalTok{] }\OperatorTok{*} \DecValTok{2}
  \ControlFlowTok{return}\NormalTok{ df_agg}
\end{Highlighting}
\end{Shaded}

\begin{Shaded}
\begin{Highlighting}[]
\NormalTok{groupby_bound(df}\OperatorTok{=}\NormalTok{df_sex_thesis,groupby_variables}\OperatorTok{=}\StringTok{'sex'}\NormalTok{,value_variables}\OperatorTok{=}\StringTok{'salary'}\NormalTok{)}
\end{Highlighting}
\end{Shaded}

\begin{verbatim}
##                mean           std   lower_bound   upper_bound
## sex                                                          
## man    14302.491879  21832.256493 -29362.021106  57967.004865
## woman  10642.502734  11949.880892 -13257.259050  34542.264519
\end{verbatim}

Now that we have a working function let's see if we can find a
difference in salary on some strata of the population

\begin{Shaded}
\begin{Highlighting}[]
\NormalTok{groupby_bound(df}\OperatorTok{=}\NormalTok{df_sex_thesis,groupby_variables}\OperatorTok{=}\NormalTok{[}\StringTok{'sex'}\NormalTok{,}\StringTok{'region'}\NormalTok{],value_variables}\OperatorTok{=}\StringTok{'salary'}\NormalTok{)}
\end{Highlighting}
\end{Shaded}

\begin{verbatim}
##                          mean           std   lower_bound   upper_bound
## sex   region                                                           
## man   midwest    15263.832776  22365.162600 -29466.492424  59994.157975
##       north      11645.335382  17755.179348 -23865.023315  47155.694079
##       south      15507.600528  24077.013334 -32646.426140  63661.627196
##       southeast  14755.538089  22471.956377 -30188.374664  59699.450842
## woman midwest    11394.125688  14039.268041 -16684.410394  39472.661770
##       north       9932.078329  10366.150119 -10800.221908  30664.378567
##       northeast  24737.436554  35861.738868 -46986.041182  96460.914290
##       south      10159.399214  10274.792359 -10390.185505  30708.983932
##       southeast  10826.777483  12140.616504 -13454.455526  35108.010491
\end{verbatim}

\begin{Shaded}
\begin{Highlighting}[]
\NormalTok{df_sex_thesis}
\end{Highlighting}
\end{Shaded}

\begin{verbatim}
##        index  age    sex  ... civil_status     region  log_salary
## 0          0   53    man  ...      married      north   11.060384
## 1          1   49  woman  ...      married      north    9.427336
## 2          2   22  woman  ...  not_married  northeast    8.378713
## 3          3   55    man  ...      married      north   11.478344
## 4          4   56  woman  ...      married      north   11.969090
## ...      ...  ...    ...  ...          ...        ...         ...
## 65790  66465   34  woman  ...      married    midwest    9.427336
## 65791  66466   40    man  ...      married    midwest    7.793999
## 65792  66467   36  woman  ...      married    midwest    7.793999
## 65793  66468   27  woman  ...      married    midwest    8.617075
## 65794  66469   37    man  ...      married    midwest    6.134157
## 
## [65795 rows x 9 columns]
\end{verbatim}

\begin{Shaded}
\begin{Highlighting}[]
\NormalTok{groupby_bound(df}\OperatorTok{=}\NormalTok{df_sex_thesis,groupby_variables}\OperatorTok{=}\NormalTok{[}\StringTok{'sex'}\NormalTok{,}\StringTok{'civil_status'}\NormalTok{],value_variables}\OperatorTok{=}\StringTok{'salary'}\NormalTok{)}
\end{Highlighting}
\end{Shaded}

\begin{verbatim}
##                             mean           std   lower_bound   upper_bound
## sex   civil_status                                                        
## man   married       16308.034145  25141.741878 -33975.449611  66591.517900
##       not_married   11345.387068  15257.186617 -19168.986167  41859.760303
## woman married       10600.663252  12113.322047 -13625.980842  34827.307346
##       not_married   10700.342305  11720.369433 -12740.396562  34141.081171
\end{verbatim}

\[f(k;p\_0^\*) = \begin{cases} p\_0^\* & \text{if }k=1, \\\\\\
1-p\_0^\* & \text {if }k=0.\end{cases}\]

Hi This is inline:
\(\mathbf{y} = \mathbf{X}\boldsymbol\beta + \boldsymbol\varepsilon\)

\end{document}
